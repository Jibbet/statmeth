\section{Sheet}
\subsection{Maximum likelihood Estimators for the multivariate Normal distribution}
The density of the multivariate normal distribution is
\begin{equation}
    f=\frac{1}{(2\pi)^{\frac{d}{2}}\sqrt{\abs{V}}}
    \exp\left(-\frac{1}{2}\left(x-\mu\right)^TV^{-1}\left(x-\mu\right)\right)
\end{equation}
The likelihood function is
\begin{align}
    g\left(\mu, V\vert x_1,\dots,x_n\right)
    &=\prod_{i=1}^{n}\frac{1}{(2\pi)^{\frac{d}{2}}\sqrt{\abs*{V}}}
    \exp\left(-\frac{1}{2}\left(x_i-\mu\right)^TV^{-1}\left(x_i-\mu\right)\right)
    \\&=(2\pi)^{-\frac{nd}{2}}\abs*{V}^{-\frac{n}{2}}
    \prod_{i=1}^{n}\exp\left(-\frac{1}{2}\left(x_i-\mu\right)^TV^{-1}\left(x_i-\mu\right)\right)
\end{align}
Calculating the log likelihood:
\begin{align}
    l = \ln g\left(\mu, V\vert x_1,\dots,x_n\right)
    &=-\frac{nd}{2}\ln\left(2\pi\right)-\frac{n}{2}\ln\left(\abs{V}\right)
    -\frac{1}{2}\sum_{i=1}^{n}\left(x_i-\mu\right)^TV^{-1}\left(x_i-\mu\right)
\end{align}
\subsubsection{Calculating $\hat{\mu}$}
\begin{align}
    \nabla_\mu \ln g &= -\frac{1}{2}\sum_{i=1}^{n}\nabla_\mu\left[\left(x_i-\mu\right)^TV^{-1}\left(x_i-\mu\right)\right]
    \\&=-\frac{1}{2}\sum_{i=1}^{n}\nabla_\mu\left[x_i^TV^{-1}x_i-x_i^TV^{-1}\mu-\mu^TV^{-1}x_i+\mu^TV^{-1}\mu\right]
    \\&=-\frac{1}{2}\sum_{i=1}^{n}\nabla_\mu\left[-x_i^TV^{-1}\mu-\mu^TV^{-1}x_i+\mu^TV^{-1}\mu\right]
    \\&=-\frac{1}{2}\sum_{i=1}^{n}\left[-x_i^TV^{-1}-v^{-1}x_i+2V^{-1}\mu\right]
    \\&=-\frac{1}{2}\sum_{i=1}^{n}2V^{-1}\left(\mu-x_i\right)
    \\&=V^{-1}\sum_{i=1}^{n}\left(x_i-\mu\right)
\end{align}
Setting this to zero yields
\begin{align}
    &\sum_{i=1}^{n}\left(x_i-\mu\right)\overset{!}{=}0
    \\\Leftrightarrow&\hat{\mu}=\frac{1}{n}\sum_{i=1}^{n}x_i
\end{align}
since $V$ is invertible.
\subsubsection{Calculating $\hat{V}$}
Use
\begin{equation}
    \abs*{V^{-1}}=\frac{1}{\abs{V}}\Rightarrow -\ln\abs{V}=\ln\abs{V^{-1}}
\end{equation}
\begin{align}
    \nabla_{V^{-1}} \ln g = \frac{n}{2}\nabla_{V^{-1}}\left[\ln\abs{V}\right]
    -\frac{1}{2}\sum_{i=1}^{n}\nabla_{V^{-1}}\left[\left(x_i-\mu\right)^TV^{-1}\left(x_i-\mu\right)\right]
\end{align}
Now we have the terms
\begin{align}
    \frac{n}{2}\nabla_{V^{-1}}\left[\ln\abs{V}\right]&=\frac{n}{2}V
\end{align}
and
\begin{align}
    \nabla_{V^{-1}}\left[\left(x_i-\mu\right)^TV^{-1}\left(x_i-\mu\right)\right]&=\nabla_{V^{-1}}\left[\trace\left(\left(x_i-\mu\right)^TV^{-1}\left(x_i-\mu\right)\right)\right]
    \\&=\nabla_{V^{-1}}\left[\trace\left(\left(x_i-\mu\right)^T\left(x_i-\mu\right)V^{-1}\right)\right]
    \\&=\left(\left(x_i-\mu\right)^T\left(x_i-\mu\right)\right)^T
    \\&=\left(x_i-\mu\right)\left(x_i-\mu\right)^T
\end{align}
The trace can be added, because the term is a scalar.
Also we used
\begin{align}
    \nabla_{B}\tr{AB}=A^T
\end{align}
Together
\begin{align}
    \nabla_{V^-1}\ln g = \frac{n}{2}V-\frac{1}{2}\sum_{i=1}^{n}\left(x_i-\mu\right)\left(x_i-\mu\right)^T\overset{!}{=}0
\end{align}
which means
\begin{equation}
    \hat{V}=\frac{1}{n}\sum_{i=1}^{n}\left(x_i-\mu\right)\left(x_i-\mu\right)^T
\end{equation}
\subsection{Radioactive source}
We are testing $H_0:\lambda\geq\frac{5}{s}$. Our sample size is $n=240$. The test statistik is $T=1182$.
\begin{align}
    P\left(T\right)&=\sum_{k=0}^T\frac{(n\lambda_0)^ke^{-n\lambda_0}}{k!}
    \\&=\sum_{k=0}^{T}\frac{1200^ke^{-1200}}{k!}\approx 0.308 \overset{!}{<}\alpha=0.05
\end{align}
This can also be thought of as chopping the 240s into 1s bits and summing to get the test statistik.
\begin{align}
    Z&=\frac{T-n\lambda_0}{\sqrt{n\lambda_0}}
    \\&=\frac{1182-1200}{\sqrt{1200}}
    \\&\approx -0.52
    \\&\approx z_{.30}
\end{align}
\subsection{Drunk intervention}
Use the t-Test for paired samples. $W_i = Y_i-X_i$. Assume $W_i$ are normally distributed.
We assume Alcohol may only increase reaction times and use a one sided t-Test.
\par$H_0:\mu_w=\mu_{w0}$
\\The test Statistic is
\begin{align}
    T &= \frac{\sqrt{n}\left(\overline{W}-\mu_{w0}\right)}{S}\approx 3.397\overset{!}{>}t_{1-\alpha;n-1}\approx 1.9
\end{align}
$H_0$ is rejected. Alcohol has an influence on reaction times.