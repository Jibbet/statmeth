\section{Sheet}
\subsection{Bad Detector}
It makes sense to assume, that if an event is only registered with
probability $p$, this can be translated to a reduced rate $\lambda_r = p\lambda$.
Therefore the Number of registered Events is poisson distributed with density
\begin{equation}
    f\left(k;\lambda_r\right)=\frac{\lambda^k}{k!}e^{-\lambda}
\end{equation}
{\color{red}yes, but rigorous?}
\subsection{Uneconomical Warranty for faulty computer monitors}
\subsubsection{Average faultless running time for economical warranty}
The percentage of failed monitors after t years is
\begin{align}
    p_f&=\int_0^t f_{\text{EX}}\left(t';\tau\right)\;dt'
    \\&= \int_0^t \frac{1}{\tau}e^{-\frac{t}{\tau}}\;dt'
    \\&=1-e^{-\frac{t}{\tau}}
\end{align}
For $t=5$ and $p_f = 0.2$:
\begin{align}
    &0.2\geq 1-e^{-\frac{5}{\tau}}
    \\\Leftrightarrow&0.8\leq e^{-\frac{5}{\tau}}
    \\\Leftrightarrow&-\frac{5}{\ln\left(0.8\right)}\leq \tau
    \\\Leftrightarrow&22.4\leq\tau
\end{align}
\subsubsection{Shortened Warranty}
\begin{equation}
    \tau\geq - \frac{3}{\ln\left(0.8\right)}=13.44
\end{equation}
\subsubsection{Monitors running after 9 years}
Assuming at least $90\%$ run after 3 years:
\begin{equation}
    \tau\geq -\frac{3}{\ln\left(0.9\right)}=28.47
\end{equation}
The percentage failed monitors after 5 years is
\begin{align}
    p_f&=1-e^{-\frac{t}{\tau}}
    \\&=0.161
\end{align}
After 5 years $0.83\%$ are still running.
\subsection{Vulcano eruption}
Let the waiting times to eruption for the $N$ identical vulcanos be realized within
the random variables $X_1, X_2, \dots, X_N$. Each one distributed exponentially
\subsubsection{Mean Time to eruption}
\subsubsection{Time until first eruption}
For each $X_i$:
\begin{align}
    &P\left(X_i\leq t\right)=\int_0^t\frac{1}{\tau}e^{-\frac{t'}{\tau}}\;dt'=1-e^{-\frac{t}{\tau}}
    \\\Rightarrow&P\left(X_i> t\right)=e^{-\frac{t}{\tau}}
\end{align}
The time until the first eruption $Y$ is now distributed as
\begin{align}
    P\left(Y\leq t\right)&=P\left(\min\left\{X_1,X_2,\dots,X_N\right\}\leq t\right)
    \\&=1-P\left(X_1>t, X_2>t,\dots,X_N>t\right)
    \\&=1-e^{-\frac{nt}{\tau}}
\end{align}
$Y$ is  distributed exponentially with mean $\frac{\tau}{n}$.
\subsubsection{Time until last eruption}
\subsection{Electric Current}
With
\begin{equation}
    \dv{R}{I}=-\frac{U}{I^2}
\end{equation}
and the Variance of $R$
\begin{align}
    &\Var\left[R\left(I\right)\right]\approx R'\left(\E\left[I\right]\right)^2\vert_{I,U}\cdot\Var\left[I\right]
    \\\Rightarrow&\Delta R= \abs{\dv{R}{I}}\Delta I
\end{align}
the variance of the resistance in linear approximation is given by
\begin{equation}
    \Delta R = \frac{230\text{V}}{2.5^2\text{A}^2}0.15\text{A}=5.52\Omega
\end{equation}
The relative error ist then
\begin{equation}
    \frac{\Delta R}{R}= 0.06
\end{equation}
\subsubsection{Different Resistor}
Halving the Resistance means halving the voltage or doubling the current.
\begin{equation}
    R' = \frac{1}{2}\frac{U}{I}
\end{equation}
Both lead to different outcomes. Assuming the voltage halves:
\begin{align}
    &\Delta R = \frac{115\text{V}}{2.5^2\text{A}^2}0.15\text{A}=2.76\Omega
    \\\Rightarrow &\frac{\Delta R}{R}=0.06
\end{align}
, because $U$ is linear in the error, as is the resistance.
