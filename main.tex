\documentclass[twoside,12pt,a4paper]{article}
\usepackage[a4paper,includeheadfoot,left=25mm,right=20mm, top=15mm, bottom=15mm]{geometry}



%---------------------------- FONT AND TYPESETTING
\usepackage[british,ngerman]{babel}		%makes latex aware of your language, so better hyphenating etc
\usepackage[T1]{fontenc} 		%use the T1 for proper searching and use of ligatures etc
\usepackage[utf8]{inputenc}		%use UTF8 encoding for reading source 
\usepackage{kpfonts}            % Schiftart hinzufügen
\usepackage{lmodern}

\usepackage{blindtext}          % Lore Ipsum einbinden
\usepackage{csquotes}           % Zitatumgebunden 

\usepackage{upgreek}            % Gerade Griechische Buchstaben

%---------------------------- PAGE LAYOUT
\usepackage[onehalfspacing]{setspace}		  %adjusts linespacing
\usepackage[font=footnotesize]{caption}   %Kleine Captions

\usepackage{lscape}                        % Querformat im Text ermöglichen
\usepackage{pdflscape}

\usepackage{float} %more figure placement

%\setlength\parindent{0pt}                 % Größe der Indents / Einrückungen im neunen Absatz

\pagestyle{plain}						  %includes page number in centred in footer

\clubpenalty = 10000                      % Schusterjungen und Huhrenkinder verbieten
\widowpenalty = 10000
\displaywidowpenalty = 10000

\usepackage{fancyhdr}                    % Hübsche Header für Seitenzahlen udn Co

\pagestyle{fancy}{                       %Define Fancy Style
\fancyhf{}
\fancyhead[RO,LE]{\thepage}
\fancyhead[RE]{\leftmark}
\fancyhead[LO]{\rightmark}
}
\pagestyle{plain}{}                     % Definiere Plaine Style

\usepackage{afterpage}
\newcommand\myemptypage{
    \null
    \thispagestyle{empty}
    \newpage
    }


%---------------------------- MATHS
\usepackage{mathtools} 						%for the rendering of maths
\usepackage{amsmath}                        % Bessere Mathematikumgebungen
\usepackage[makeroom]{cancel}               %crossing out terms
\DeclareMathOperator{\sinc}{sinc}           %Why do i need this :(
\numberwithin{equation}{section}			%reset numbering within a structural object
\def\Var{{\textrm{Var}}\,}
\def\E{{\textrm{E}}\,}
\usepackage{mathpazo}                       % Mathematik Font
\usepackage{upgreek}
\usepackage{faktor}                         %for faktor sets/spaces
\usepackage{tikz}                           %
\usetikzlibrary{tikzmark}                   %for overset arrows

%---------------------------- SCIENCE
\usepackage{units}                          % Setzen von Einheiten
\usepackage{physics}
\usepackage{chemfig}

%---------------------------- FIGURES AND TABLES
\usepackage{caption} 
\usepackage{subcaption}
\usepackage{sidecap}
\usepackage[export]{adjustbox} 
\usepackage{graphicx}					    %for the rendering of floating graphics
\usepackage{sidecap}
\usepackage{color, colortbl}

\usepackage[section]{placeins}				%defines float barriers at the end of sections. Set option [section] for this.
\usepackage{array} 							%for creating tables
\usepackage{booktabs}						%professional looking tables, provides /toprule etc
\usepackage{tabularx}
\usepackage{pdfpages}



%---------------------------- BIBLIOGRAPHY
\usepackage[backend=biber,style=numeric-comp,hyperref, url=false, eprint=false, sorting=none]{biblatex}

\addbibresource{./Bibliographie.bib}	%the absolute or relative path of your bibliography file.
%\usepackage[nottoc,numbib]{tocbibind}

\usepackage{hyperref, bookmark}			%turns the references and citations into hyperlinks. This needs to be last on the preamble.
\hypersetup{colorlinks=true,linkcolor=black,citecolor=black,urlcolor=black}


\setlength{\bibitemsep}{.2\baselineskip plus .05\baselineskip minus .05\baselineskip}

\DefineBibliographyStrings{ngerman}{ 
   andothers = {et\addabbrvspace al\adddot},
   andmore   = {et\addabbrvspace al\adddot},
}
%\renewcommand*{\bibfont}{\small}

%---------------------------- Misc
\usepackage{xcolor}
\usepackage{todonotes}

%%%%%%%%%%%%%%%%%%%%%%%%%%%%%%%%%%%%%%%%%%%%%%%%%%%%%%%%


\begin{document}

%	Put the document stuff in here!
\fontfamily{phv}\selectfont

\pagestyle{empty}


\begin{titlepage}
	\begin{center}
	\vspace{0.5cm}
    {\scshape\LARGE Statistische Methoden der Datenanalyse\par}
	\vspace{2cm}
	 \rule{16.5cm}{.5pt} \\\vspace{0.5cm}
	{\Huge\bfseries Excersise Sheets\par}
	\vspace{0.5cm}
    \rule{16.5cm}{.5pt} \\

	\vspace{2cm}
	
\large  \hfill \\ 	\medskip


\bigskip\medskip



    \vspace{4cm}
	\vfill
	
    {\Large Physik \\
    	\bigskip
    \LARGE\scshape	Uni Wien \par}
    
    \bigskip

    \vfill
	{\large \today\par}
	\end{center}
\end{titlepage}


\fontfamily{ptm}\selectfont

\clearpage

 \myemptypage 

\pagestyle{fancy}
\clearpage\setcounter{figure}{0}
\setlength{\tabcolsep}{0pt}
\makeatletter 
\let\c@table\c@figure
\let\c@lstlisting\c@figure
\renewcommand*\figurename{}
\renewcommand*\tablename{}
\newpage
%\section{Sheet}
\subsection{Introduction}
\subsubsection{Minimum Value of $P\left(A\cap B\right)$}
\begin{align}
    &P\left(A\cup B\right)=P\left(A\right)+P\left(B\right)-P\left(A\cap B\right)
    \\\Leftrightarrow &P\left(A\cap B\right)=P\left(A\right)+P\left(B\right)-P\left(A\cup B\right)
\end{align}
To minimize $P\left(A\cup B\right)$ maximize $P\left(A\cup B\right)$!
Since $P\left(A\right)+P\left(B\right)\geq1$ we can assume that ${A\cup B = \Omega}$ in effort to maximize
$P\left(A\cup B\right)$. Therefore set $P\left(A\cup B\right) = 1$. Now
\begin{equation}
    \min_{A,B}P\left(A\cap B\right)=P\left(A\right)+P\left(B\right)-P\left(A\cup B\right)=\frac{1}{21}
\end{equation}
\subsubsection{A and B independent}
For $A$ and $B$ independent
\begin{equation}
    P\left(A\cap B\right)=P\left(A\right)P\left(B\right)=\frac{5}{21}
\end{equation}
\subsubsection{C:= None of the Events occour}
The Negations of Events are independent for independent Events. So,
\begin{equation}
    P\left(A'\cap B'\right)=P\left(A'\right)P\left(B'\right)=\frac{2}{3}\frac{2}{7}=\frac{4}{21}
\end{equation}
\subsubsection{D:= Exactly one of the two events occours}
Lets define $AB:=A\cap B$.
\begin{align}
    P\left(A'B\cup AB'\right)&=P\left(A'B\right)+P\left(AB'\right)-P\left(\underbrace{A'B\cap AB'}_{=\emptyset}\right)\label{disjointIntersection}
    \\&=P\left(A'B\right)+P\left(AB'\right)
    \\&=P\left(A'\right)P\left(B\right)+P\left(A\right)P\left(B'\right)
    \\&=\frac{2}{3}\frac{5}{7}+\frac{1}{3}\frac{2}{7}=\frac{4}{7}
\end{align}
\subsubsection{E:= Both Events occour}
For $A$ and $B$ independent
\begin{equation}
    P\left(A\cap B\right)=P\left(A\right)P\left(B\right)=\frac{1}{3}\frac{5}{7}=\frac{5}{21}
\end{equation}
\paragraph{Sanity Check}
Since the already calculated Events $C:=\textbf{No Event occuring}$, $D:=\textbf{One Event occuring}$ and $E:=\textbf{Both events occuring}$ span $\Omega$, we can check that
\begin{equation}
    P\left(C\right)+P\left(D\right)+P\left(E\right)=\frac{4}{21}+\frac{4}{7}+\frac{5}{21}\overset{!}{=}1
\end{equation}
and indeed it is.
\subsubsection{F:= At least one of the two Events occours}
\paragraph{Fast Route}
Since $F=D\dot\cup E$
\begin{equation}
    P\left(F\right)=P\left(D\right)+P\left(E\right)=\frac{17}{21}
\end{equation}
\paragraph{Long, but instructive Route}
Again, using $AB:=A\cap B$,
\begin{align}
    P\left(D\cup E\right)&=P\left(A'B\cup AB'\cup AB\right)
    \\&=P\left(A'B\right)+P\left(AB'\cup AB\right)-P\left(\underbrace{A'B\cap\left(AB'\cup AB\right)}_{=\emptyset}\right)\label{disjointIntersection2}
    \\&=P\left(A'B\right)+P\left(AB'\right)+P\left(AB\right)-P\left(\underbrace{AB'\cap AB}_{=\emptyset}\right)\label{disjointIntersection3}
    \\&=P\left(A'\right)P\left(B\right)+P\left(A\right)P\left(B'\right)+P\left(A\right)P\left(B\right)
    \\&=\frac{2}{3}\frac{5}{7}+\frac{1}{3}\frac{2}{7}+\frac{1}{3}\frac{5}{7}
    \\&=\frac{17}{21}
\end{align}
\subsubsection{G:= At most one of the two Events occours}
\paragraph{Fast Route}
Since $G:=C\dot\cup D$
\begin{equation}
    P\left(G\right)=P\left(C\right)+P\left(D\right)=\frac{16}{21}
\end{equation}
\paragraph{Long (shortened), but instructive Route}
Using the same empty Intsection argument as in \ref{disjointIntersection}, \ref{disjointIntersection2} and \ref{disjointIntersection3}
\begin{align}
    P\left(A'B'\cup A'B\cup AB'\right)&=P\left(A'B'\right)+P\left(A'B\right)+P\left(AB'\right)
    \\&=\frac{2}{3}\frac{2}{7}+\frac{2}{3}\frac{5}{7}+\frac{1}{3}\frac{2}{7}=\frac{16}{21}
\end{align}
\subsection{Novel Detector Failure}
There are two remoTES, each fully functional ($c$) with a probability
\begin{equation}
    P\left(c\right) = P\left(p_1\right)P\left(w\right)P\left(p_2\right) = 0.405
\end{equation},
since the sucess events of three components ($p_1$, $w$ and $p_2$) are independent.
Each one will fail with a probability
\begin{equation}
    P\left(\overline{c}\right)=1-0.405=0.595
\end{equation}
The probability for both to fail is
\begin{equation}
    P\left(\overline{c_1}\cap \overline{c_2}\right) = P^2\left(\overline{c}\right)=0.354025
\end{equation}
Which means at least one will function with probability
\begin{equation}
    P\left(c_1\cup c_2\right) = 1-0.354025= 0.645975
\end{equation}
\subsection{graduation rate}
\subsubsection{graduation rate}
With the Events $w:=$ \textbf{is Woman}, $g:=$ \textbf{graduated} the probability of degree completion is
\begin{align}
    P\left(g\right)&=P\left(g\vert w\right)P\left(w\right)+P\left(g\vert \overline{w}\right)P\left(\overline{w}\right)
    \\&= 0.372\cdot 0.625+0.311\cdot \left(1-0.625\right)=0.349125
\end{align}
\subsubsection{percentage of male dropouts}
Since we now know the unconditional dropout rate, we can calculate
\begin{align}
    P\left(m\vert \overline{g}\right) &= \frac{P\left(m\cap \overline{g}\right)}{P\left(\overline{g}\right)}
    \\&=\frac{P\left(\overline{g}\vert m\right)P\left(m\right)}{P\left(\overline{g}\right)}
    \\&=\frac{\left(1-0.311\right)\left(1-0.625\right)}{1-0.349125}
    \\&\approx 0.397
\end{align}

%\section{Sheet}
\subsection{Bad Detector}
It makes sense to assume, that if an event is only registered with
probability $p$, this can be translated to a reduced rate $\lambda_r = p\lambda$.
Therefore the Number of registered Events is poisson distributed with density
\begin{equation}
    f\left(k;\lambda_r\right)=\frac{\lambda^k}{k!}e^{-\lambda}
\end{equation}
{\color{red}yes, but rigorous?}
\subsection{Uneconomical Warranty for faulty computer monitors}
\subsubsection{Average faultless running time for economical warranty}
The percentage of failed monitors after t years is
\begin{align}
    p_f&=\int_0^t f_{\text{EX}}\left(t';\tau\right)\;dt'
    \\&= \int_0^t \frac{1}{\tau}e^{-\frac{t}{\tau}}\;dt'
    \\&=1-e^{-\frac{t}{\tau}}
\end{align}
For $t=5$ and $p_f = 0.2$:
\begin{align}
    &0.2\geq 1-e^{-\frac{5}{\tau}}
    \\\Leftrightarrow&0.8\leq e^{-\frac{5}{\tau}}
    \\\Leftrightarrow&-\frac{5}{\ln\left(0.8\right)}\leq \tau
    \\\Leftrightarrow&22.4\leq\tau
\end{align}
\subsubsection{Shortened Warranty}
\begin{equation}
    \tau\geq - \frac{3}{\ln\left(0.8\right)}=13.44
\end{equation}
\subsubsection{Monitors running after 9 years}
Assuming at least $90\%$ run after 3 years:
\begin{equation}
    \tau\geq -\frac{3}{\ln\left(0.9\right)}=28.47
\end{equation}
The percentage failed monitors after 5 years is
\begin{align}
    p_f&=1-e^{-\frac{t}{\tau}}
    \\&=0.161
\end{align}
After 5 years $0.83\%$ are still running.
\subsection{Vulcano eruption}
Let the waiting times to eruption for the $N$ identical vulcanos be realized within
the random variables $X_1, X_2, \dots, X_N$. Each one distributed exponentially
\subsubsection{Mean Time to eruption}
\subsubsection{Time until first eruption}
For each $X_i$:
\begin{align}
    &P\left(X_i\leq t\right)=\int_0^t\frac{1}{\tau}e^{-\frac{t'}{\tau}}\;dt'=1-e^{-\frac{t}{\tau}}
    \\\Rightarrow&P\left(X_i> t\right)=e^{-\frac{t}{\tau}}
\end{align}
The time until the first eruption $Y$ is now distributed as
\begin{align}
    P\left(Y\leq t\right)&=P\left(\min\left\{X_1,X_2,\dots,X_N\right\}\leq t\right)
    \\&=1-P\left(X_1>t, X_2>t,\dots,X_N>t\right)
    \\&=1-e^{-\frac{nt}{\tau}}
\end{align}
$Y$ is  distributed exponentially with mean $\frac{\tau}{n}$.
\subsubsection{Time until last eruption}
\subsection{Electric Current}
The variance of the resistance in linear approximation is given by
\begin{align}
    \Var\left[R\left(I\right)\right]&\approx R'\left(\E\left[I\right]\right)^2\vert_{I,U}\cdot\Var\left[I\right]
    \\&= \left(-\frac{230\text{V}}{2.5\text{A}}\right)^2\cdot\left(0.15\text{A}\right)^2
    \\&= 190.44 A^2
    \\\Rightarrow \Delta R = 13.8\Omega
\end{align}
The relative error ist then
\begin{equation}
    \frac{\Delta R}{R}= \frac{13.8 \text{A}}{\frac{230\text{V}}{2.5\text{A}}}=0.18\Omega
\end{equation}
\subsubsection{Different Resistor}
The Resistor is $R=92\Omega \Rightarrow R'=46\Omega$. Yes? It changes to $0.15\Omega$, assuming the voltage changes and the current stays constant.

%\section{Sheet}
\subsection*{Inverse Distribution}
Let $X$ be gamma distributed $\text{Ga}\left(a,b\right),a>2$ and $Y = 1/x$.
\subsubsection{Density of $1/X$}
\paragraph{Short Route}
Since $Y=h\left(X\right)=1/X\Longrightarrow X=h^{-1}\left(Y\right)=\frac{1}{Y}$
\begin{equation}
    f_Y(y)=\frac{1}{X^2}f_X\left(\frac{1}{X}\right)
\end{equation}
\paragraph{Long Route}
The Distribution function of $Y=1/X$ can be written as
\begin{equation}
    P\left(\frac{1}{X}\leq x\right)=P\left(\frac{1}{x}\leq X\right)=1-P\left(X<\frac{1}{x}\right)=1-F_{\text{Ga}}\left(\frac{1}{x};a,b\right)
\end{equation}
From this the density can be retrieved by derivating with respect to $y$:
\begin{equation}
    \pdv{x}\left[1-F_{\text{Ga}}\left(\frac{1}{x};a,b\right)\right]=-\pdv{x}\left[F_{\text{Ga}}\left(\frac{1}{x};a,b\right)\right]=\frac{1}{x^2}f_{\text{Ga}}\left(\frac{1}{x};a,b\right)
\end{equation}
Note that no properties of the Gamma distribution have been used and this is a general result.
\subsubsection{expectation value of $1/X$}
The expectation value is
\begin{align}
    E\left[\frac{1}{X}\right]&=\int_{-\infty}^{\infty}\frac{1}{x}f_{\text{Ga}}\left(x;a,b\right)\;dx
    \\&=\int_0^\infty \frac{1}{x}\frac{x^{a-1}e^{-x/b}}{b^a\Gamma\left(a\right)}\;dx
    \\&=\int_0^\infty \frac{x^{a-2}e^{-x/b}}{bb^a\left(a-1\right)\Gamma\left(a-1\right)}\;dx
    \\&=\frac{1}{b\left(a-1\right)}\int_{-\infty}^\infty f_\text{Ga}\left(x;a-1,b\right)\;dx
    \\&=\frac{1}{b\left(a-1\right)}
\end{align}
\subsubsection{variance of $1/X$}
The variance is
\begin{align}
    V\left[\frac{1}{X}\right]&=E\left[\frac{1}{X^2}\right]-E\left[\frac{1}{X}\right]^2
    \\&=\frac{1}{b^2\left(a-2\right)\left(a-1\right)}-\frac{1}{b^2\left(a-1\right)^2}
    \\&=\frac{\left(a-1\right)}{b^2\left(a-2\right)\left(a-1\right)^2}-\frac{\left(a-2\right)}{b^2\left(a-2\right)\left(a-1\right)^2}
    \\&=\frac{1}{b^2\left(a-2\right)\left(a-1\right)^2}
\end{align}
\paragraph{Linear Error propagation}
\begin{align}
    \frac{1}{X}\approx\frac{1}{\mu}-\frac{1}{\mu^2}\left(x-\mu\right)+\frac{1}{\mu^3}\left(x-\mu\right)^2
\end{align}
\begin{align}
    E\left[\frac{1}{X}\right]&\approx \frac{1}{\mu}+\frac{1}{2}\frac{2}{\mu^2}\cdot V\left[X\right]
    \\&=\frac{1}{ab}+\frac{1}{a^2b^2}\cdot ab^2
    \\&=\frac{1}{ab}+\frac{1}{a}
\end{align}
\subsection{Energy spectrum}
\subsubsection{convolution}
\begin{align}
    f_m&=\frac{1}{10}\left[9f_{No}\left(\mu=5.9,\sigma_0^2\right)+f_{No}\left(\mu = 6.49,\sigma_1^2\right)\right]\star f_{No}\left(\mu = 0, \sigma_F^2\right)
    \\&=\frac{9}{10}\left[f_{No}\left(\mu=5.9,\sigma_0^2\right)\star f_{No}\left(\mu = 0, \sigma_F^2\right)\right]
    \\&+\frac{1}{10}\left[f_{No}\left(\mu=6.49,\sigma_1^2\right)\star f_{No}\left(\mu = 0, \sigma_F^2\right)\right]
    \\&=\frac{9}{10}f_{No}\left(\mu=5.9,\sigma_0^2+\sigma_F^2\right)+\frac{1}{10}f_{No}\left(\mu=6.49,\sigma_1^2+\sigma_F^2\right)
\end{align}
\subsubsection{natural uncertainties}
\begin{align}
    &\sigma^2 = \tilde{\sigma}^2-\sigma_F^2 = 0.15 \text{keV}^2 \Rightarrow \sigma = 0.39 \text{keV}
\end{align}

%\section{Sheet}
\subsection{ML Estimator}
\subsubsection{Calculating the Estimator}
The Joint probability density of an exponentially distributed sample is
\begin{equation}
    g\left(x_1,\dots,x_m\vert \tau\right) =\prod_{i=1}^{n}\frac{1}{\tau}e^{-\frac{x_i}{\tau}}
\end{equation}
The logarithm of this density interpreted as a likelihood function is
\begin{align}
    l\left(\tau\right)=\ln g\left(x_1,\dots,x_n\vert \tau\right)&=\sum_{i=1}^n \ln\left(\frac{1}{\tau}e^{-\frac{x_i}{\tau}}\right)
    \\&=\sum_{i=1}^n -\ln\left(\tau\right)-\frac{1}{\tau}x_i
\end{align}
Maximizing the chance to draw this particular sample:
\begin{align}
    &\pdv{l}{\tau}=\sum_{i=1}^n -\frac{1}{\tau}+\frac{1}{\tau^2}x_i \overset{!}{=} 0
    \\\Leftrightarrow&-n +\sum_{i=1}^n\frac{1}{\tau}x_i = 0 
    \\\Leftrightarrow&\hat{\tau} = \frac{1}{n}\sum^n_{i=1}x_i
\end{align}
Inserting $s=\sum x_i = 395.25$ and $n=250$ yields
\begin{equation}
    \hat{\tau} = 1.581
\end{equation}
\subsubsection{Showing that the Estimator is efficient}
\paragraph{Showing $\hat{\tau}$ is unbiased}
\begin{align}
    E_\tau\left[\frac{1}{n}\sum_{i=1}^nx_i\right]&=\frac{1}{n}\sum_{i=1}^n E\left[x_i\right]
    \\&=\frac{1}{n}\sum_{i=1}^n\tau
    \\&=\tau
\end{align}
\paragraph{Showing $\hat{\tau}$ is efficient}
Calculate the Fisher Information
\begin{align}
    I_\tau &= E\left[-\pdv[2]{\ln g\left(x_1,\dots,x_2\vert \tau\right)}{\tau}\right]
    \\&=E\left[-\sum_{i=1}^n \frac{1}{\tau^2}-\frac{2}{\tau^3}x_i\right]
    \\&=-\sum_{i=1}^n \frac{1}{\tau^2}-\frac{2}{\tau^3}E\left[x_i\right]
    \\&=-\sum_{i=1}^n \frac{1}{\tau^2}-\frac{2}{\tau^2}
    \\&=\sum_{i=1}^n \frac{1}{\tau^2}
    \\&=\frac{n}{\tau^2}
\end{align}
\paragraph{Comparing with estimator variance}
\begin{align}
    V\left[\hat{\tau}\right] &= V\left[\frac{1}{n^2}\sum_{i=1}^n V\left[x_i\right]\right]
    \\&=\frac{\tau^2}{n}
\end{align}
Conclusion, estimator is as efficient as it gets.
\subsection{Laplace distribution}
\subsubsection{Expectation Value and Variance}
The Laplace distribution density is
\begin{align}
    f\left(x;m,s\right)=\frac{1}{2s}\exp\left[-\frac{\abs{x-m}}{s}\right]
\end{align}
\paragraph{Expectation Value}
The expecation value of laplace distributed variable $X$ is
\begin{align}
    E\left[X\right]&=\int_{-\infty}^{\infty}\frac{x}{2s}\exp\left[-\frac{\abs*{x-m}}{s}\right]\;dx
    \\&=\frac{1}{2s}\int_{-\infty}^\infty \left(u+m\right)\exp\left[-\frac{\abs{u}}{s}\right]\;du
    \\&=\frac{1}{2s}\int_{-\infty}^\infty u\exp\left[-\frac{\abs{u}}{s}\right]\;du+m
    \\&=m
\end{align}
\paragraph{Variance}
\begin{align}
    V\left[X\right]&=E\left[\left(X-m\right)^2\right]
    \\&=\frac{1}{2s}\int_{-\infty}^{\infty}\left(x-m\right)^2\exp{-\frac{\abs{x-m}}{s}}\;dx
    \\&=\frac{1}{2s}\int_{-\infty}^{\infty} u^2\exp{-\frac{\abs*{u}}{s}}\;du
    \\&=\frac{1}{s}\int_0^\infty u^2\exp{-\frac{u}{s}}\;du
    \\&=s^3\int_0^\infty v^2 e^{-v}\;dv
    \\&=2s^3
\end{align}
\subsubsection{Estimators for $m$ and $s$}
For a given sample the joint density is
\begin{align}
    g\left(x_1,\dots,x_n\vert s,m\right)=\prod_{i=1}^{n}\frac{1}{2s}e^{-\frac{\abs*{x_i-m}}{s}}
\end{align}
The log likelihood function is
\begin{align}
    \ln g &= \sum_{i=1}^n \ln\left(\frac{1}{2s}\exp{-\frac{\abs{x_i-m}}{s}}\right)
    \\&=\sum_{i=1}^{n}\ln\left(\frac{1}{2s}\right)-\frac{\abs{x_i-m}}{s}
    \\&=n\ln\left(\frac{1}{2s}\right)-\frac{1}{s}\sum_{i=1}^n\abs{x_i-m}
\end{align}
The maximum likelihood estimator for $m$ can now be found
\begin{align}
    \pdv{\ln g}{s}&=\pdv{s}\left[n\ln\left(\frac{1}{2s}\right)-\frac{1}{s}\sum_{i=1}^n\abs{x_i-m}\right]
    \\&=-\frac{n}{s}+\frac{1}{s^2}\sum_{i=1}^n\abs{x_i-m}\overset{!}{=}0
    \\\Rightarrow \hat{s}&=\frac{1}{n}\sum_{i=1}^n\abs{x_i-m}
\end{align}
The maximum likelihood estimator for $s$ can also be found
\begin{align}
    \pdv{\ln g}{m}&=\pdv{m}\left[n\ln\left(\frac{1}{2s}\right)-\frac{1}{s}\sum_{i=1}^n\abs{x_i-m}\right]
    \\&=-\frac{1}{s}\sum_{i=1}^n\pdv{m}\abs{x_i-m}
    \\&=-\frac{1}{s}\sum_{i=1}^n \text{sgn}\left(x-m\right)
\end{align}
\subsection{Survey}
The multimodal distribution is defined as
\begin{align}
    f\left(n_A,n_B,n_C,n_D\vert p_A,p_B,p_C,p_D\right)=n!\prod_{i=A,B,C,D}\frac{1}{n_i!}p_i^{n_i}
\end{align}
with the log density
\begin{align}
    \ln f &= \ln\left(n!\prod_{i=A,B,C,D}\frac{1}{n_i!}p_{i}^{n_i}\right)
    \\&=\ln\left(n!\right)+\sum_{i=A,B,C,D}\ln\left(\frac{1}{n_i!}\right)+\ln\left(p_i^{n_i}\right)
    \\&=\ln\left(n!\right)+\sum_{i=A,B,C,D}\ln\left(\frac{1}{n_i!}\right)+n_i\ln\left(p_i\right)
\end{align}
Now the estimators for the voter shares can be found with
\begin{align}
    \pdv{\ln f}{p_i}&=\frac{n_i}{p_i}
\end{align}

\section{Sheet}
\subsection{Bernoulli}
\subsubsection{Clopper and Pearson confidence interval}
A Bernoulli experiment is repeated $n=200$ times with $k=121$ sucesses. Calculate the symmetric $95\%$ interval
for the  parameter $p$.
The Interval boundaries can be calculated with the inverse beta distribution.
\begin{align}
    G_1\left(k\right)&=\beta\left(\frac{\alpha}{2};k,n-k+1\right)=0.534
    \\G_2\left(k\right)&=\beta\left(\frac{1-\alpha}{2};k+1,n-k\right)=0.673
\end{align}
\subsubsection{Approximation by normal distribution (bootstrap and robust)}
Estimate $p$:
\begin{equation}
    \hat{p}=\frac{k}{n}=\frac{121}{200}
\end{equation}
With that estimate $\sigma$
\begin{align}
    \sigma\left[\hat{p}\right]&=\sqrt{\frac{\hat{p}\left(1-\hat{p}\right)}{n}}
    \\&=\sqrt{\frac{9559}{8\cdot 10^6}}\approx 0.035
\end{align}
\begin{equation}
    z_{1-\frac{\alpha}{2}}=f_{\text{norm}}\left(1-\frac{\alpha}{2}\right)=0.248
\end{equation}
Now the interval boundaries are for the bootstrap method:
\begin{align}
    G_1\left(k\right)&=\hat{p}-z_{1-\frac{\alpha}{2}}\sqrt{\frac{\hat{p}\left(1-\hat{p}\right)}{n}}\approx 0.591
    \\G_1\left(k\right)&=\hat{p}+z_{1-\frac{\alpha}{2}}\sqrt{\frac{\hat{p}\left(1-\hat{p}\right)}{n}}\approx 0.619
\end{align}
and for the robust method
\begin{align}
    G_1\left(k\right)&=\hat{p}-z_{1-\frac{\alpha}{2}}\frac{1}{2\sqrt{n}}\approx 0.585
    \\G_1\left(k\right)&=\hat{p}+z_{1-\frac{\alpha}{2}}\frac{1}{2\sqrt{n}}\approx 0.625
\end{align}
\subsubsection{Agresti-Coull}
\begin{align}
    G_1\left(k\right)\approx 0.585
    \\G_2\left(k\right)\approx 0.624
\end{align}
\subsection{Biased and unbiased Estimators for uniform distribution interval borders}
The joint probability of the sample ist
\begin{equation}
    g\left(X_1,\dots,X_N\vert a,b\right)=\prod_n^N \frac{1}{b-a}\cdot I_{\left[a,b\right]}\left(X_n\right)
\end{equation}
The ML estimator is
\begin{equation}
    \hat{a}=\text{arg max}_a\prod_{n}^{N}\frac{1}{b-a}\cdot I_{\left[a,b\right]}\left(X_n\right)
\end{equation}
This would not take a minimum if not for the constraint
\begin{equation}
    \hat{a}\leq\min_n\left\{X_n\right\}
\end{equation}
Therefore
\begin{equation}
    \hat{a}=\min_n\left\{X_n\right\}
\end{equation}
Similarly
\begin{equation}
    \hat{b}\geq\max_n\left\{X_n\right\}
\end{equation}
and
\begin{equation}
    \hat{b}=\max_n\left\{X_n\right\}
\end{equation}
\paragraph{Showing the estimators are biased}
To show the estimators are biased, calculate their distribution functions:
\begin{align}
    F_{\hat{a}}\left(x\right)=P\left(\hat{a}\leq x\right)&=1-\prod_n P\left(X_n>x\right)
    \\&=1-P^N\left(X>x\right)
    \\&=1-\left(\frac{b-x}{b-a}\right)^N
\end{align}
The density is
\begin{align}
    \pdv{F_{\hat{a}}}{x}&=\frac{N}{b-a}\left(\frac{b-x}{b-a}\right)^{N-1}
\end{align}
Now the expectation value is
\begin{align}
    E\left(\hat{a}\right)&=\int_a^b x \frac{n}{b-a}\left(\frac{b-x}{b-a}\right)^{N-1}\;dx
    \\&=\left[-\left(\frac{b-x}{b-a}\right)^N x\right]_a^b+\int_a^b\left(\frac{b-x}{b-a}\right)^N\;dx
\end{align}
Here is
\begin{align}
    \left[-\left(\frac{b-x}{b-a}\right)^N x\right]_a^b=\left[-\underbrace{\left(\frac{b-b}{b-a}\right)^N}_{=0}b + \underbrace{\left(\frac{b-a}{b-a}\right)^N}_{=1}a\right]=a
\end{align}
and
\begin{align}
    \int_a^b\left(\frac{b-x}{b-a}\right)^N\;dx&=\left[-\frac{b-a}{N+1}\left(\frac{b-x}{b-a}\right)^{N+1}\right]_a^b
    \\&=-\frac{b-a}{N+1}\left[\underbrace{\left(\frac{b-b}{b-a}\right)^{N+1}}_{=0}-\underbrace{\left(\frac{b-a}{b-a}\right)^{N+1}}_{=1}\right]
    \\&=\frac{b-a}{N+1}
\end{align}
Together
\begin{align}
    E\left(\hat{a}\right)=a+\frac{b-a}{N+1}
\end{align}
Similarly for $\hat{b}$:
\begin{align}
    F_{\hat{b}}\left(x\right)&=P\left(\hat{b}\leq x\right)
    \\&=\prod_n P\left(X_n\leq x\right)
    \\&=P^N\left(X\leq x\right)
    \\&=\left(\frac{x-a}{b-a}\right)^N
\end{align}
The density is
\begin{equation}
    \pdv{F_{\hat{b}}}{x}=\frac{N}{b-a}\left(\frac{x-a}{b-a}\right)^{N-1}
\end{equation}
The expectation value of $\hat{b}$ is
\begin{align}
    E(\hat{b})&=\int_{a}^{b}x\frac{N}{b-a}\left(\frac{x-a}{b-a}\right)^{N-1}\;dx
    \\&=\left[\left(\frac{x-a}{b-a}\right)^N x\right]_a^b-\int_{a}^{b}\left(\frac{x-a}{b-a}\right)^N\;dx
\end{align}
where
\begin{align}
    \left[\left(\frac{x-a}{b-a}\right)^N x\right]_a^b&=\left[\underbrace{\left(\frac{b-a}{b-a}\right)^N}_{=1}b-\underbrace{\left(\frac{a-a}{b-a}\right)^N}_{=0}a\right]=b
\end{align}
and
\begin{align}
    \int_{a}^{b}\left(\frac{x-a}{b-a}\right)^N\;dx&=\left[\frac{b-a}{N+1}\left(\frac{x-a}{b-a}\right)^{N+1}\right]_a^b
    \\&=\frac{b-a}{N+1}\left[\underbrace{\left(\frac{b-a}{b-a}\right)^{N+1}}_{=1}-\underbrace{\left(\frac{a-a}{b-a}\right)^{N+1}}_{=0}\right]
    \\&=\frac{b-a}{N+1}
\end{align}
Together
\begin{align}
    E(\hat{b})=b-\frac{b-a}{N+1}
\end{align}
\paragraph{Conclusion} Both estimators are biased, but asymptotically unbiased. This makes intuitivly sense.
We would like to correct the estimators $\hat{a}$ and $\hat{b}$.
\begin{align}
    \hat{a}_c &= \min_n\left\{X_n\right\}-\frac{b-a}{N+1}
    \\\hat{b}_c &= \max_n\left\{X_n\right\}+\frac{b-a}{N+1}
\end{align}
but $a$ and $b$ are not a priori known.
Note that
\begin{equation}
    \frac{E(\hat{a}+\hat{b})}{2}=\frac{E(\hat{a})+E(\hat{b})}{2}=\frac{a+b}{2}
\end{equation}
is an estimator for the mean and unbiased.
\end{document}