\documentclass[twoside,12pt,a4paper]{article}
\usepackage[a4paper,includeheadfoot,left=25mm,right=20mm, top=15mm, bottom=15mm]{geometry}



%---------------------------- FONT AND TYPESETTING
\usepackage[british,ngerman]{babel}		%makes latex aware of your language, so better hyphenating etc
\usepackage[T1]{fontenc} 		%use the T1 for proper searching and use of ligatures etc
\usepackage[utf8]{inputenc}		%use UTF8 encoding for reading source 
\usepackage{kpfonts}            % Schiftart hinzufügen
\usepackage{lmodern}

\usepackage{blindtext}          % Lore Ipsum einbinden
\usepackage{csquotes}           % Zitatumgebunden 

\usepackage{upgreek}            % Gerade Griechische Buchstaben

%---------------------------- PAGE LAYOUT
\usepackage[onehalfspacing]{setspace}		  %adjusts linespacing
\usepackage[font=footnotesize]{caption}   %Kleine Captions

\usepackage{lscape}                        % Querformat im Text ermöglichen
\usepackage{pdflscape}

\usepackage{float} %more figure placement

%\setlength\parindent{0pt}                 % Größe der Indents / Einrückungen im neunen Absatz

\pagestyle{plain}						  %includes page number in centred in footer

\clubpenalty = 10000                      % Schusterjungen und Huhrenkinder verbieten
\widowpenalty = 10000
\displaywidowpenalty = 10000

\usepackage{fancyhdr}                    % Hübsche Header für Seitenzahlen udn Co

\pagestyle{fancy}{                       %Define Fancy Style
\fancyhf{}
\fancyhead[RO,LE]{\thepage}
\fancyhead[RE]{\leftmark}
\fancyhead[LO]{\rightmark}
}
\pagestyle{plain}{}                     % Definiere Plaine Style

\usepackage{afterpage}
\newcommand\myemptypage{
    \null
    \thispagestyle{empty}
    \newpage
    }


%---------------------------- MATHS
\usepackage{mathtools} 						%for the rendering of maths
\usepackage{amsmath}                        % Bessere Mathematikumgebungen
\usepackage[makeroom]{cancel}               %crossing out terms
\DeclareMathOperator{\sinc}{sinc}           %Why do i need this :(
\numberwithin{equation}{section}			%reset numbering within a structural object
\def\Var{{\textrm{Var}}\,}
\def\E{{\textrm{E}}\,}
\usepackage{mathpazo}                       % Mathematik Font
\usepackage{upgreek}
\usepackage{faktor}                         %for faktor sets/spaces
\usepackage{tikz}                           %
\usetikzlibrary{tikzmark}                   %for overset arrows

%---------------------------- SCIENCE
\usepackage{units}                          % Setzen von Einheiten
\usepackage{physics}
\usepackage{chemfig}

%---------------------------- FIGURES AND TABLES
\usepackage{caption} 
\usepackage{subcaption}
\usepackage{sidecap}
\usepackage[export]{adjustbox} 
\usepackage{graphicx}					    %for the rendering of floating graphics
\usepackage{sidecap}
\usepackage{color, colortbl}

\usepackage[section]{placeins}				%defines float barriers at the end of sections. Set option [section] for this.
\usepackage{array} 							%for creating tables
\usepackage{booktabs}						%professional looking tables, provides /toprule etc
\usepackage{tabularx}
\usepackage{pdfpages}



%---------------------------- BIBLIOGRAPHY
\usepackage[backend=biber,style=numeric-comp,hyperref, url=false, eprint=false, sorting=none]{biblatex}

\addbibresource{./Bibliographie.bib}	%the absolute or relative path of your bibliography file.
%\usepackage[nottoc,numbib]{tocbibind}

\usepackage{hyperref, bookmark}			%turns the references and citations into hyperlinks. This needs to be last on the preamble.
\hypersetup{colorlinks=true,linkcolor=black,citecolor=black,urlcolor=black}


\setlength{\bibitemsep}{.2\baselineskip plus .05\baselineskip minus .05\baselineskip}

\DefineBibliographyStrings{ngerman}{ 
   andothers = {et\addabbrvspace al\adddot},
   andmore   = {et\addabbrvspace al\adddot},
}
%\renewcommand*{\bibfont}{\small}

%---------------------------- Misc
\usepackage{xcolor}
\usepackage{todonotes}

%%%%%%%%%%%%%%%%%%%%%%%%%%%%%%%%%%%%%%%%%%%%%%%%%%%%%%%%


\begin{document}

%	Put the document stuff in here!
\fontfamily{phv}\selectfont

\pagestyle{empty}


\begin{titlepage}
	\begin{center}
	\vspace{0.5cm}
    {\scshape\LARGE Statistische Methoden der Datenanalyse\par}
	\vspace{2cm}
	 \rule{16.5cm}{.5pt} \\\vspace{0.5cm}
	{\Huge\bfseries Excersise Sheets\par}
	\vspace{0.5cm}
    \rule{16.5cm}{.5pt} \\

	\vspace{2cm}
	
\large  \hfill \\ 	\medskip


\bigskip\medskip



    \vspace{4cm}
	\vfill
	
    {\Large Physik \\
    	\bigskip
    \LARGE\scshape	Uni Wien \par}
    
    \bigskip

    \vfill
	{\large \today\par}
	\end{center}
\end{titlepage}


\fontfamily{ptm}\selectfont

\clearpage

 \myemptypage 

\pagestyle{fancy}
\clearpage\setcounter{figure}{0}
\setlength{\tabcolsep}{0pt}
\makeatletter 
\let\c@table\c@figure
\let\c@lstlisting\c@figure
\renewcommand*\figurename{}
\renewcommand*\tablename{}
\newpage
\section{Sheet}
\subsection{Introduction}
\subsubsection{Minimum Value of $P\left(A\cap B\right)$}
\begin{align}
    &P\left(A\cup B\right)=P\left(A\right)+P\left(B\right)-P\left(A\cap B\right)
    \\\Leftrightarrow &P\left(A\cap B\right)=P\left(A\right)+P\left(B\right)-P\left(A\cup B\right)
\end{align}
To minimize $P\left(A\cup B\right)$ maximize $P\left(A\cup B\right)$!
Since $P\left(A\right)+P\left(B\right)\geq1$ we can assume that ${A\cup B = \Omega}$ in effort to maximize
$P\left(A\cup B\right)$. Therefore set $P\left(A\cup B\right) = 1$. Now
\begin{equation}
    \min_{A,B}P\left(A\cap B\right)=P\left(A\right)+P\left(B\right)-P\left(A\cup B\right)=\frac{1}{21}
\end{equation}
\subsubsection{A and B independent}
For $A$ and $B$ independent
\begin{equation}
    P\left(A\cap B\right)=P\left(A\right)P\left(B\right)=\frac{5}{21}
\end{equation}
\subsubsection{C:= None of the Events occour}
The Negations of Events are independent for independent Events. So,
\begin{equation}
    P\left(A'\cap B'\right)=P\left(A'\right)P\left(B'\right)=\frac{2}{3}\frac{2}{7}=\frac{4}{21}
\end{equation}
\subsubsection{D:= Exactly one of the two events occours}
Lets define $AB:=A\cap B$.
\begin{align}
    P\left(A'B\cup AB'\right)&=P\left(A'B\right)+P\left(AB'\right)-P\left(\underbrace{A'B\cap AB'}_{=\emptyset}\right)\label{disjointIntersection}
    \\&=P\left(A'B\right)+P\left(AB'\right)
    \\&=P\left(A'\right)P\left(B\right)+P\left(A\right)P\left(B'\right)
    \\&=\frac{2}{3}\frac{5}{7}+\frac{1}{3}\frac{2}{7}=\frac{4}{7}
\end{align}
\subsubsection{E:= Both Events occour}
For $A$ and $B$ independent
\begin{equation}
    P\left(A\cap B\right)=P\left(A\right)P\left(B\right)=\frac{1}{3}\frac{5}{7}=\frac{5}{21}
\end{equation}
\paragraph{Sanity Check}
Since the already calculated Events $C:=\textbf{No Event occuring}$, $D:=\textbf{One Event occuring}$ and $E:=\textbf{Both events occuring}$ span $\Omega$, we can check that
\begin{equation}
    P\left(C\right)+P\left(D\right)+P\left(E\right)=\frac{4}{21}+\frac{4}{7}+\frac{5}{21}\overset{!}{=}1
\end{equation}
and indeed it is.
\subsubsection{F:= At least one of the two Events occours}
\paragraph{Fast Route}
Since $F=D\dot\cup E$
\begin{equation}
    P\left(F\right)=P\left(D\right)+P\left(E\right)=\frac{17}{21}
\end{equation}
\paragraph{Long, but instructive Route}
Again, using $AB:=A\cap B$,
\begin{align}
    P\left(D\cup E\right)&=P\left(A'B\cup AB'\cup AB\right)
    \\&=P\left(A'B\right)+P\left(AB'\cup AB\right)-P\left(\underbrace{A'B\cap\left(AB'\cup AB\right)}_{=\emptyset}\right)\label{disjointIntersection2}
    \\&=P\left(A'B\right)+P\left(AB'\right)+P\left(AB\right)-P\left(\underbrace{AB'\cap AB}_{=\emptyset}\right)\label{disjointIntersection3}
    \\&=P\left(A'\right)P\left(B\right)+P\left(A\right)P\left(B'\right)+P\left(A\right)P\left(B\right)
    \\&=\frac{2}{3}\frac{5}{7}+\frac{1}{3}\frac{2}{7}+\frac{1}{3}\frac{5}{7}
    \\&=\frac{17}{21}
\end{align}
\subsubsection{G:= At most one of the two Events occours}
\paragraph{Fast Route}
Since $G:=C\dot\cup D$
\begin{equation}
    P\left(G\right)=P\left(C\right)+P\left(D\right)=\frac{16}{21}
\end{equation}
\paragraph{Long (shortened), but instructive Route}
Using the same empty Intsection argument as in \ref{disjointIntersection}, \ref{disjointIntersection2} and \ref{disjointIntersection3}
\begin{align}
    P\left(A'B'\cup A'B\cup AB'\right)&=P\left(A'B'\right)+P\left(A'B\right)+P\left(AB'\right)
    \\&=\frac{2}{3}\frac{2}{7}+\frac{2}{3}\frac{5}{7}+\frac{1}{3}\frac{2}{7}=\frac{16}{21}
\end{align}
\subsection{Novel Detector Failure}
There are two remoTES, each fully functional ($c$) with a probability
\begin{equation}
    P\left(c\right) = P\left(p_1\right)P\left(w\right)P\left(p_2\right) = 0.405
\end{equation},
since the sucess events of three components ($p_1$, $w$ and $p_2$) are independent.
Each one will fail with a probability
\begin{equation}
    P\left(\overline{c}\right)=1-0.405=0.595
\end{equation}
The probability for both to fail is
\begin{equation}
    P\left(\overline{c_1}\cap \overline{c_2}\right) = P^2\left(\overline{c}\right)=0.354025
\end{equation}
Which means at least one will function with probability
\begin{equation}
    P\left(c_1\cup c_2\right) = 1-0.354025= 0.645975
\end{equation}
\subsection{graduation rate}
\subsubsection{graduation rate}
With the Events $w:=$ \textbf{is Woman}, $g:=$ \textbf{graduated} the probability of degree completion is
\begin{align}
    P\left(g\right)&=P\left(g\vert w\right)P\left(w\right)+P\left(g\vert \overline{w}\right)P\left(\overline{w}\right)
    \\&= 0.372\cdot 0.625+0.311\cdot \left(1-0.625\right)=0.349125
\end{align}
\subsubsection{percentage of male dropouts}
Since we now know the unconditional dropout rate, we can calculate
\begin{align}
    P\left(m\vert \overline{g}\right) &= \frac{P\left(m\cap \overline{g}\right)}{P\left(\overline{g}\right)}
    \\&=\frac{P\left(\overline{g}\vert m\right)P\left(m\right)}{P\left(\overline{g}\right)}
    \\&=\frac{\left(1-0.311\right)\left(1-0.625\right)}{1-0.349125}
    \\&\approx 0.397
\end{align}

\section{Sheet}
\subsection{Bad Detector}
It makes sense to assume, that if an event is only registered with
probability $p$, this can be translated to a reduced rate $\lambda_r = p\lambda$.
Therefore the Number of registered Events is poisson distributed with density
\begin{equation}
    f\left(k;\lambda_r\right)=\frac{\lambda^k}{k!}e^{-\lambda}
\end{equation}
{\color{red}yes, but rigorous?}
\subsection{Uneconomical Warranty for faulty computer monitors}
\subsubsection{Average faultless running time for economical warranty}
The percentage of failed monitors after t years is
\begin{align}
    p_f&=\int_0^t f_{\text{EX}}\left(t';\tau\right)\;dt'
    \\&= \int_0^t \frac{1}{\tau}e^{-\frac{t}{\tau}}\;dt'
    \\&=1-e^{-\frac{t}{\tau}}
\end{align}
For $t=5$ and $p_f = 0.2$:
\begin{align}
    &0.2\geq 1-e^{-\frac{5}{\tau}}
    \\\Leftrightarrow&0.8\leq e^{-\frac{5}{\tau}}
    \\\Leftrightarrow&-\frac{5}{\ln\left(0.8\right)}\leq \tau
    \\\Leftrightarrow&22.4\leq\tau
\end{align}
\subsubsection{Shortened Warranty}
\begin{equation}
    \tau\geq - \frac{3}{\ln\left(0.8\right)}=13.44
\end{equation}
\subsubsection{Monitors running after 9 years}
Assuming at least $90\%$ run after 3 years:
\begin{equation}
    \tau\geq -\frac{3}{\ln\left(0.9\right)}=28.47
\end{equation}
The percentage failed monitors after 5 years is
\begin{align}
    p_f&=1-e^{-\frac{t}{\tau}}
    \\&=0.161
\end{align}
After 5 years $0.83\%$ are still running.
\subsection{Vulcano eruption}
Let the waiting times to eruption for the $N$ identical vulcanos be realized within
the random variables $X_1, X_2, \dots, X_N$. Each one distributed exponentially
\subsubsection{Mean Time to eruption}
\subsubsection{Time until first eruption}
For each $X_i$:
\begin{align}
    &P\left(X_i\leq t\right)=\int_0^t\frac{1}{\tau}e^{-\frac{t'}{\tau}}\;dt'=1-e^{-\frac{t}{\tau}}
    \\\Rightarrow&P\left(X_i> t\right)=e^{-\frac{t}{\tau}}
\end{align}
The time until the first eruption $Y$ is now distributed as
\begin{align}
    P\left(Y\leq t\right)&=P\left(\min\left\{X_1,X_2,\dots,X_N\right\}\leq t\right)
    \\&=1-P\left(X_1>t, X_2>t,\dots,X_N>t\right)
    \\&=1-e^{-\frac{nt}{\tau}}
\end{align}
$Y$ is  distributed exponentially with mean $\frac{\tau}{n}$.
\subsubsection{Time until last eruption}
\subsection{Electric Current}
The variance of the resistance in linear approximation is given by
\begin{align}
    \Var\left[R\left(I\right)\right]&\approx R'\left(\E\left[I\right]\right)^2\vert_{I,U}\cdot\Var\left[I\right]
    \\&= \left(-\frac{230\text{V}}{2.5\text{A}}\right)^2\cdot\left(0.15\text{A}\right)^2
    \\&= 190.44 A^2
    \\\Rightarrow \Delta R = 13.8\Omega
\end{align}
The relative error ist then
\begin{equation}
    \frac{\Delta R}{R}= \frac{13.8 \text{A}}{\frac{230\text{V}}{2.5\text{A}}}=0.18\Omega
\end{equation}
\subsubsection{Different Resistor}
The Resistor is $R=92\Omega \Rightarrow R'=46\Omega$. Yes? It changes to $0.15\Omega$, assuming the voltage changes and the current stays constant.

\end{document}