\section{Sheet}
\subsection{Introduction}
\subsubsection{Minimum Value of $P\left(A\cap B\right)$}
\begin{align}
    &P\left(A\cup B\right)=P\left(A\right)+P\left(B\right)-P\left(A\cap B\right)
    \\\Leftrightarrow &P\left(A\cap B\right)=P\left(A\right)+P\left(B\right)-P\left(A\cup B\right)
\end{align}
To minimize $P\left(A\cup B\right)$ maximize $P\left(A\cup B\right)$!
Since $P\left(A\right)+P\left(B\right)\geq1$ we can assume that ${A\cup B = \Omega}$ in effort to maximize
$P\left(A\cup B\right)$. Therefore set $P\left(A\cup B\right) = 1$. Now
\begin{equation}
    \min_{A,B}P\left(A\cap B\right)=P\left(A\right)+P\left(B\right)-P\left(A\cup B\right)=\frac{1}{21}
\end{equation}
\subsubsection{A and B independent}
For $A$ and $B$ independent
\begin{equation}
    P\left(A\cap B\right)=P\left(A\right)P\left(B\right)=\frac{5}{21}
\end{equation}
\subsubsection{C:= None of the Events occour}
The Negations of Events are independent for independent Events. So,
\begin{equation}
    P\left(A'\cap B'\right)=P\left(A'\right)P\left(B'\right)=\frac{2}{3}\frac{2}{7}=\frac{4}{21}
\end{equation}
\subsubsection{D:= Exactly one of the two events occours}
Lets define $AB:=A\cap B$.
\begin{align}
    P\left(A'B\cup AB'\right)&=P\left(A'B\right)+P\left(AB'\right)-P\left(\underbrace{A'B\cap AB'}_{=\emptyset}\right)\label{disjointIntersection}
    \\&=P\left(A'B\right)+P\left(AB'\right)
    \\&=P\left(A'\right)P\left(B\right)+P\left(A\right)P\left(B'\right)
    \\&=\frac{2}{3}\frac{5}{7}+\frac{1}{3}\frac{2}{7}=\frac{4}{7}
\end{align}
\subsubsection{E:= Both Events occour}
For $A$ and $B$ independent
\begin{equation}
    P\left(A\cap B\right)=P\left(A\right)P\left(B\right)=\frac{1}{3}\frac{5}{7}=\frac{5}{21}
\end{equation}
\paragraph{Sanity Check}
Since the already calculated Events $C:=\textbf{No Event occuring}$, $D:=\textbf{One Event occuring}$ and $E:=\textbf{Both events occuring}$ span $\Omega$, we can check that
\begin{equation}
    P\left(C\right)+P\left(D\right)+P\left(E\right)=\frac{4}{21}+\frac{4}{7}+\frac{5}{21}\overset{!}{=}1
\end{equation}
and indeed it is.
\subsubsection{F:= At least one of the two Events occours}
\paragraph{Fast Route}
Since $F=D\dot\cup E$
\begin{equation}
    P\left(F\right)=P\left(D\right)+P\left(E\right)=\frac{17}{21}
\end{equation}
\paragraph{Long, but instructive Route}
Again, using $AB:=A\cap B$,
\begin{align}
    P\left(D\cup E\right)&=P\left(A'B\cup AB'\cup AB\right)
    \\&=P\left(A'B\right)+P\left(AB'\cup AB\right)-P\left(\underbrace{A'B\cap\left(AB'\cup AB\right)}_{=\emptyset}\right)\label{disjointIntersection2}
    \\&=P\left(A'B\right)+P\left(AB'\right)+P\left(AB\right)-P\left(\underbrace{AB'\cap AB}_{=\emptyset}\right)\label{disjointIntersection3}
    \\&=P\left(A'\right)P\left(B\right)+P\left(A\right)P\left(B'\right)+P\left(A\right)P\left(B\right)
    \\&=\frac{2}{3}\frac{5}{7}+\frac{1}{3}\frac{2}{7}+\frac{1}{3}\frac{5}{7}
    \\&=\frac{17}{21}
\end{align}
\subsubsection{G:= At most one of the two Events occours}
\paragraph{Fast Route}
Since $G:=C\dot\cup D$
\begin{equation}
    P\left(G\right)=P\left(C\right)+P\left(D\right)=\frac{16}{21}
\end{equation}
\paragraph{Long (shortened), but instructive Route}
Using the same empty Intsection argument as in \ref{disjointIntersection}, \ref{disjointIntersection2} and \ref{disjointIntersection3}
\begin{align}
    P\left(A'B'\cup A'B\cup AB'\right)&=P\left(A'B'\right)+P\left(A'B\right)+P\left(AB'\right)
    \\&=\frac{2}{3}\frac{2}{7}+\frac{2}{3}\frac{5}{7}+\frac{1}{3}\frac{2}{7}=\frac{16}{21}
\end{align}
\subsection{Novel Detector Failure}
There are two remoTES, each fully functional ($c$) with a probability
\begin{equation}
    P\left(c\right) = P\left(p_1\right)P\left(w\right)P\left(p_2\right) = 0.405
\end{equation},
since the sucess events of three components ($p_1$, $w$ and $p_2$) are independent.
Each one will fail with a probability
\begin{equation}
    P\left(\overline{c}\right)=1-0.405=0.595
\end{equation}
The probability for both to fail is
\begin{equation}
    P\left(\overline{c_1}\cap \overline{c_2}\right) = P^2\left(\overline{c}\right)=0.354025
\end{equation}
Which means at least one will function with probability
\begin{equation}
    P\left(c_1\cup c_2\right) = 1-0.354025= 0.645975
\end{equation}
\subsection{graduation rate}
\subsubsection{graduation rate}
With the Events $w:=$ \textbf{is Woman}, $g:=$ \textbf{graduated} the probability of degree completion is
\begin{align}
    P\left(g\right)&=P\left(g\vert w\right)P\left(w\right)+P\left(g\vert \overline{w}\right)P\left(\overline{w}\right)
    \\&= 0.372\cdot 0.625+0.311\cdot \left(1-0.625\right)=0.349125
\end{align}
\subsubsection{percentage of male dropouts}
Since we now know the unconditional dropout rate, we can calculate
\begin{align}
    P\left(m\vert \overline{g}\right) &= \frac{P\left(m\cap \overline{g}\right)}{P\left(\overline{g}\right)}
    \\&=\frac{P\left(\overline{g}\vert m\right)P\left(m\right)}{P\left(\overline{g}\right)}
    \\&=\frac{\left(1-0.311\right)\left(1-0.625\right)}{1-0.349125}
    \\&\approx 0.397
\end{align}
